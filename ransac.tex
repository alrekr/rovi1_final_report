%!TEX root = mainfile.tex
\subsection{RANSAC and \textsc{SURF}}
%!TEX root = mainfile.tex
\begin{figure}
\centering
\begin{tikzpicture}[node distance=2cm,scale=0.6, every node/.style={transform shape}]
    \node (start) [startstop] {Start};
    \node (objectin) [io, below of=start] {Marker in};
    \node (detectobject) [process, below of=objectin] {Detect keypoints};
    \node (computeobject) [process, below of=detectobject] {Compute descriptors};

    \node (perimage) [startstop, left of=start, xshift=12cm] {Per image};
    \node (imgin) [io, below of=perimage] {Image in};
    \node (detect) [process, below of=imgin] {Detect keypoints};
    \node (compute) [process, below of=detect] {Compute descriptors};
    \node (match) [process, below of=compute] {Match descriptors with marker};
    \node (dist) [process, below of=match] {Calculate minimum distance between two descriptors, use for filtering away very bad matches};
    \node (homo) [process, below of=dist] {Calculate homography};
    \node (trans) [process, below of=homo] {Perform perspective transform};
    \node (extract) [process, below of=trans] {Extract center point using marker corners};
    \node (stop) [startstop, below of=extract] {Stop};

    \draw[arrow] (start) -- (objectin);
    \draw[arrow] (objectin) -- (detectobject);
    \draw[arrow] (detectobject) -- (computeobject);
    \draw[arrow, dashed] (computeobject) |- node[anchor=north] {Make descriptors available for} (match);

    \draw[arrow] (perimage) -- (imgin);
    \draw[arrow] (imgin) -- (detect);
    \draw[arrow] (detect) -- (compute);
    \draw[arrow] (compute) -- (match);
    \draw[arrow] (match) -- (dist);
    \draw[arrow] (dist) -- (homo);
    \draw[arrow] (homo) -- (trans);
    \draw[arrow] (trans) -- (extract);
    \draw[arrow] (extract) -- (stop);
\end{tikzpicture}
\caption{Loading of image and finding the corny marker and center.}
\label{fig:corny_flowchart}
\end{figure}

Speeded up robust features, \textsc{SURF}, is faster than \textsc{SIFT} according to its authors\cite{WikiSURF}, thus it is better in a time-critical implementation.
It is also claimed to be more robust that \textsc{SIFT}, and so it is a better choice than SIFT.\@
There appear to be no difference in how to implement either in OpenCV.\@
The process for finding the marker in an image is shown in Figure~\vref{fig:corny_flowchart}.

First, when the object of the \verb|Ransac| class is constructed the Corny marker is loaded and preliminary computations are done.
Then, when each new image is first assigned to the object and the center is extracted, the following happens:
\begin{enumerate}
    \item Find keypoints are found and descriptors computed
    \item Match descriptors for the true marker and the image using Flann matcherd
    \item Calculate distance between matches
    \begin{enumerate}
    \item Throw away all matches which are less than three times minimum distance
    \end{enumerate}
    \item Get coordinates for good matches
    \item Calculate homography based on coordinates
    \item Find corners in true marker
    \item Perform perspective transform
    \item Find corners in marker in image
    \item Find center of the image using the average of the corners
\end{enumerate}

Using that approach, the center of the image as well as the four corners are available.

It should now be clear that even though the two feature extractors are very different in their construction and method,
they come to much the same result; four points in a quadrilateral and a point in the center of mass.
Both methods are called the same way; instantiate the class, \verb|assign(Mat)| an image and \verb|extract()| the features wanted.

To extract the center point: \verb|Point2f p = class.extract();|.
To extract three points: \verb|Point2f p1, p2, p3; class.extract(p1, p2, p3);|.
