%!TEX root = mainfile.tex
\subsection{Inverse kinematics}
In order to use the image information, it is necessary to perform inverse kinematics, and to convert the image coordinate to coordinates for the robot.
The end result should be

\begin{align}
\mathrm{\textbf{dq}}=\left[\textbf{Z}_\mathit{image}(\textbf{q})\right]^T\textbf{y} % chktex 3
\end{align}

Going backwards, $\textbf{Z}_\mathit{image}(\textbf{q}) = \textbf{J}_\mathit{image}\textbf{S}(\textbf{q})\textbf{J}(\textbf{q})$. Here,

\begin{align}
    \textbf{J}_\mathit{image}&=
    \left[
        \begin{matrix}
            -\frac{f}{z} & 0            & \frac{u}{z} & \frac{uv}{f}      & -\frac{f^2+u^2}{f} & v \\
            0            & -\frac{f}{z} & \frac{v}{z} & \frac{f^2+v^2}{f} & -\frac{uv}{f}      & -u
        \end{matrix}
    \right]\label{eq:J1}\\
    \textbf{S}(\textbf{q})&=\left[
    \begin{matrix}
        \left[\textbf{R}_\mathit{base}^\mathit{tool}(\textbf{q})\right]^T & | & 0 \\ % chktex 3
        -- -- & & -- -- \\
        0 & | & \left[\textbf{R}_\mathit{base}^\mathit{tool}(\textbf{q})\right]^T % chktex 3
    \end{matrix}
    \right]\label{eq:J2}\\
    \textbf{J}(\textbf{q})&=\left[
    \begin{matrix}
        \textbf{A}(\textbf{q})\\
        ------\\
        \textbf{B}(\textbf{q})
    \end{matrix}
    \right]\label{eq:J3}
\end{align}

where, for Equation~\ref{eq:J1}, $f$ is the camera focal length and

\begin{align}
u &= \frac{fx}{z}\nonumber \\
v &= \frac{fy}{z}
\end{align}

It is important to note that the $x$ and $y$ values do not correspond to pixel values from OpenCV, but must be offset with respectively half image width and half image height.

Equations~\ref{eq:J2} and~\ref{eq:J3} are further explored in~\cite{robbook}.
Suffice to say, they are based on the robot's position $\textbf{p}$ and state vector $\textbf{q}$ among others.
